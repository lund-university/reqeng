\begin{tabular}{| p{2.3cm} |p{6.2cm} | p{6.2cm} | p{6.2cm} |}
\hline
{\it Assessment area} & {\it Required for project grade {\bf 3: Accepable}} \newline Demonstrate ability to ... & {\it Also required for project grade {\bf 4: Good}} \newline Demonstrate ability to ...& {\it Also required for project grade {\bf 5: Excellent}} \newline Demonstrate ability to ... \\
\hline
\hline
{\bf Specification} &
%3
    {\bf 3A)} apply more than one suitable specification technique (e.g. task descriptions and screen prototypes), and more than two types of requirement (e.g. data, function, quality), and more than three abstraction levels (e.g. goal, domain, product, design). \newline
    {\bf 3B)} define a system's boundaries and its interaction with external entities. \newline
    {\bf 3C)} reflect on specification experiences and reason about choices of specification methods in relation to different contexts. &
%4
    {\bf 4A)} combine different degrees of completeness and different levels of abstraction. \newline
    {\bf 4B)} use at least four different specification techniques adequately tailored to the context. \newline
    {\bf 4C)} provide explicit requirements rationale that reduce risks of misinterpretation. \newline
    {\bf 4D)} use hierarchies and requirements relations to manage evolving requirements structures. &
%5
    {\bf 5A)} combine specification techniques in an explicitly motivated trade-off between qualities and costs, where a high degree of specification completeness is achieved for a carefully selected subset of requirements.     \newline
    {\bf 5B)} provide motivated estimations of target quality levels using well-defined scales.
\\ \hline

{\bf Elicitation}  &
    {\bf 3D)} apply more than one elicitation technique in a relevant way. \newline
    {\bf 3E)} reflect on elicitation experiences. &

    {\bf 4E)} reason about the need for further elicitation in relation to specification quality. \newline
     {\bf 4F)} demonstrate good use of prototyping to elicit realistic user requirements.&
   

    {\bf 5C)} go beyond initial stakeholders and given frames, while challenging the domain boundaries and eliciting creative ideas and deep domain knowledge in real-world contexts.
\\ \hline

{\bf Validation}  &

    {\bf 3F)} to assess the quality of requirements and find relevant  problems of several different types. \newline
    {\bf 3G)} apply more than one validation technique including prototyping. \newline
    {\bf 3H)} reflect on validation experiences.&

    {\bf 4G)} to find, prioritize and discuss requirements quality problems of different types, while reaching beyond form issues. \newline
    {\bf 4H)} adapt the validation to the context and provide rationale for the chosen validation techniques. &

    {\bf 5D)} reason about the relation between requirements quality problems and risks, both from a product owner and developer viewpoint. \newline
    {\bf 5E)} utilize links among different types of specifications in validation efforts to find and address potentially harmful inconsistencies. \newline
\\ \hline

{\bf Selection}  &
  {\bf 3I)} use more than one prioritization technique in a relevant way. \newline
  {\bf 3J)} reflect on prioritization experiences.
&
  {\bf 4I)} create a release plan for a subset of prioritized features, while taking into account precedence constraints. \newline
&
  {\bf 5F)} combine priorities from several stakeholders and use priorities and scheduling constraints to iteratively create a relevant release plan.\newline
  {\bf 5G)} use prioritization to focus improvements of specification quality and elicitation efforts for a well-motivated subset of requirements.
\\ \hline

\end{tabular}
