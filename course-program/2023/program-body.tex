%!TEX root = program-ETSN15.tex

\maketitle
\noindent 
The objective of the course is to give basic and advanced knowledge and skills within requirements engineering for large-scale development of systems completely or partly based on software. The course gives both theoretical knowledge and practical skills in methods and techniques for requirements engineering. The course gives training in scientific paper reading.

\section{Learning Objectives}
\subsection{Knowledge and understanding}
For a passing grade the student must
\begin{enumerate}[noitemsep]
\item be able to define basic concepts and principles within requirements engineering 
\item give an account of several different types of requirements
\item be able to describe and value several different methods and techniques for requirements engineering
\item be able to describe and relate different sub-processes within requirements engineering
\item be able to describe the relation between the requirements engineering process and other processes in the product lifecycle
\ifteknolog
	\item be able to describe the relation between requirements engineering and market-driven product management
	\item be able to discuss some scientific results within requirements engineering research
\fi
\end{enumerate}


\subsection{Skills and abilities}
For a passing grade the student must
\begin{enumerate}[noitemsep]
\item be able to choose suitable requirements techniques for a given context
\item     be able to apply several different techniques for requirements elicitation
\item     be able to apply several different techniques for requirements specification
\item     be able to apply several different techniques for requirements validation
\item     be able to apply several different techniques for requirements prioritisation
\end{enumerate}
\ifteknolog
  The release plan defines which requirements that are implemented by the project group as mock-up designs in release R3, and which requirements are selected to be fully implemented in the imagined releases R4 and R5. 
 \fi
\subsection{Judgement and approach}
For a passing grade the student must
\begin{enumerate}[noitemsep]
\item     be able to consciously select a process depending on the nature of the requirements
\item     show a systematic and long-term approach to processes
\item     be able to consciously see the problem in the relation between the quality of requirements and the quality of the resulting implementation
\item     be able to adequately involve users in the requirements engineering process
\ifteknolog
	\item     be able to consciously see the problem in the relation between requirements engineering and economical aspects of product development
\fi
\end{enumerate}

\section{Contents}
The course includes theory and practice regarding the following topics:
\begin{enumerate}[noitemsep]
\item Requirements on different abstraction levels and in different contexts

\item Sub-processes of requirements engineering and their relation

\item Specification of data requirements, e.g. using virtual windows and data models

\item Specification of functional requirements, e.g. using textual feature requirements and task descriptions

\item Specification of different types of non-functional requirements, e.g. usability, performance, reliability

\item Different techniques for requirements elicitation

\item Different techniques for requirements validation

\item Different techniques for requirements prioritization

\ifteknolog
	\item Market-driven requirements engineering and product management
\fi
\end{enumerate}

\newpage

\section{Course elements}
\begin{description}
\item[L: Lectures] The lectures provide an overview of the literature. They do not cover every detail, but give a high-level structure of the subject and thereby aid self-studies of the literature. Discussions are promoted.
\item[E: Exercises] The main objective of the exercises is to support the project and prepare for the written exam through prototypical problems, by connecting theory to practice and to give opportunity to discuss details of RE techniques.
\item[LAB: Computer lab sessions] The lab sessions illustrate computer supported
prioritization and release planning, and demonstrates the complexity of requirements selection and scheduling. Preparations are mandatory. %
\ifteknolog\else
The release planning part is optional.
\fi
\ifteknolog
	\item[P: Project] The project involves performing practical requirements engineering for a given case, and is carried out in groups of 6-8 students. The project involves a number of deliverables and a final project conference where the learning outcome of each project is presented. Project groups are established during the first course week.
\else
	\item[P: Project] The project involves performing practical requirements engineering for a given case, and is performed individually. The project involves a number of deliverables.
\fi

\end{description}




\section{Assessment}
\begin{itemize}
\item The project is graded fail | 3 | 4 | 5 based on project deliverables.
\item Approved lab session preparations and assignments are required for passing.
\item The written exam comprises two parts: a multple-choice part and a part with essay and practical assignments. 50 percent is required for each part for passing. The essay and practical assignment part yields max 100 points and determines the grade fail | 3 | 4 | 5.
\item The final course grade on the scale fail | 3 | 4 | 5 is based on the written exam points (from essay and practical assignment) and the project grade using the following mapping: 

\begin{tabular}{r | c c c}
 & Project: 3 & Project: 4 & Project: 5 \\
\hline
 & \multicolumn{3}{c}{Exam points}    \\
Final: 3 & $ \geq 50$ & $\geq 50$ & $\geq 50$ \\
Final: 4 & $ \geq 75$ & $\geq 67$ & $\geq 60$ \\
Final: 5 & $ \geq 90$ & $\geq 83$ & $\geq 75$ \\
\hline
\end{tabular}


\end{itemize}

\ifteknolog\else
\newpage
\fi

\section{Literature}
The course elements and the written exam will cover the following literature: 
\begin{flushleft}
\setlength{\tabcolsep}{0pt}
\begin{tabular}{p{0.15\columnwidth} p{0.85\columnwidth}}
Lau & Soren Lauesen, Software Requirements - Styles and Techniques, Addison-Wesley, ISBN 0-201-74570-4, 2002. \\
LAB1\&2	&Preparations and instructions for Lab 1: ''Requirements Modeling'' and Lab 2: ''Requirements Prioritization and Release Planning'' \\
\ifteknolog
	MDRE &	''Market-Driven Requirements Engineering for Software Products'', Björn Regnell and Sjaak Brinkkemper, Engineering and Managing Software Requirements, Eds. A. Aurum and C. Wohlin, Springer,  ISBN 3-540-25043-3, 2005 \\
\fi
PRIO&	''Requirements Prioritization'', Patrik Berander and Anneliese Andrews, Engineering and Managing Software Requirements, Eds. A. Aurum and C. Wohlin, Springer,  ISBN 3-540-25043-3, 2005 \\
\ifteknolog
	OSSRE & ''Understanding Requirements for Open Source Software'', Walt Scacchi, Design Requirements Engineering: A Ten-Year Perspective, Springer, pp. 467-494, 2009\\
 RP&	''The Art and Science of Software Release Planning'', Günther Ruhe and Moshood Omolade Saliu, IEEE Software, November/December, pp. 47-53, 2005  \\
\fi
QUPER&	Supporting Roadmapping of Quality Requirements - B Regnell, Richard Berntsson Svensson, 
Thomas Olsson, IEEE Software 25(2) pp 42-47 March-April 2008  \\
PROTO1& ''A Model of Software Prototyping based on a Systematic Map'', Bjarnason, E., Lang, F., Mjoberg, A., International Symposium on Empirical Software Engineering and Measurement (ESEM). 2021 \\
PROTO2& ''Prototyping practices in software startups: Initial case study results'', Bjarnason, E., IEEE 29th International Requirements Engineering Conference Workshops (REW). 2021\\
INSP&	''Att inspektera krav''. Sid 67-76, Framgångsrik kravhantering, andra utgåvan, Teknikföretagen, Joachim Karlsson, V040072, ISSN 1103-7008, 1998\\
\ifteknolog
	 INTDEP &	''An industrial survey of requirements interdependencies in software product release planning'', Carlshamre, P., Sandahl, K., Lindvall, M., Regnell, B., Natt och Dag, J.: Int. Conf. on Requirements Engineering (RE01), Toronto, Canada, pp. 84–91, 2001 \\
\fi
AGRE &	''Agile Requirements Engineering Practices: An Empirical Study'', Lan Cao, Balasubramaniam Ramesh, IEEE Software , January/February 2008, pp.60-67, 2008 \\
\end{tabular}
\end{flushleft}

\ifteknolog\else
\noindent See what lecture L1-L8 is connected to which literature item on the next page.
\fi

\newpage

\section{Overview}

%% IT IS NOT ALLOWED TO HAVE CONDITIONALS INSIDE tabular
%% so adding this newcommand, which is interpreted before the below tabular
\ifteknolog
\newcommand{\CONF}{
& P & Conference presentation (pdf) & & Deadline Wed 8:00  \\
& L8 & Project conference &  & Wed 15:05-17\\
}
\else
\newcommand{\CONF}{}
\fi

\ifteknolog
\newcommand{\LTWOPROJECT}{
& L2 & Meet your Product Owner, Elicit, Prio  & Lau:8, PRIO  & Wed 15-17\\
}
\else
\newcommand{\LTWOPROJECT}{
& L2 & Project kick-off, Elicit, Prio  & Lau:8, PRIO  & Wed 15-17\\
}
\fi

\ifteknolog
\newcommand{\LASTSUPERVISION}{book with supervisor on \newline Mon or Tue (or W6) }
\else
\newcommand{\LASTSUPERVISION}{Mon 13}
\fi

\begin{flushleft}
\small
\begin{tabular}{c | p{0.6cm} p{4.4cm} p{2.2cm}  p{3.1cm}}
 &  & {\it Topic} & {\it Literature} & {\it When}   \\
\hline
\multirow{4}{*}{{\bfseries\sffamily W1}} 
& L1& Introduction  & Lau:1 & Tue 15-17\\
\LTWOPROJECT
& E1 & Requirements types, \newline Context diagram &  Lau:1  & Thu 10-12 or 13-15\\
\hline
\multirow{4}{*}{{\bfseries\sffamily W2}} 
& L3& Specification 1, reqT    & Lau: 2-4  & Tue 15-17\\
& L4& Specification 2  &  & Wed 15-17 \\
& P  & Project Mission&  & Tue 23.59 \\
& E2 & Elicitation  & Lau: 8  & Thu 10-12 or 13-15\\
& Lab1 & Reqts Modelling, Prio & LAB1 &   We 15-17, Fr 10-12 or 13-15\\
& P & Supervision meeting & & \\
\hline
\multirow{4}{*}{{\bfseries\sffamily W3}} 

& L5~EB& Agile RE, \newline Prototyping,  
\newline Validation & AGRE, PROTO1-2, Lau:9, INSP & Tue 15-17 \\

& \STARTFRG{}L6~JL & Product mgmt, Release planning, Open source RE, Interdep. & MDRE, RP, \newline OSSRE, \newline INTDEP & Wed 15-17 \\
& E3 & Functional requirements  & Lau:2-4  & Thu 10-12 or 13-15\\
& P & Release R1& & Deadline Sun 23.59 \\
\hline
\multirow{3}{*}{{\bfseries\sffamily W4}} 
& P & Supervision meeting & & \\
& L7 & Specification 3, Quality, Lifecycle & Lau:5-7, QUPER  & Tu 15-17\\
& E4 & Quality requirements &  Lau:6, QUPER  &Thu 10-12 or 13-15\\
& Lab2 & Prioritization, Release Planning & LAB2 &   We 15-17, Fr 10-12 or 13-15\\
\hline
\multirow{3}{*}{{\bfseries\sffamily W5}} 
%& L8 & Validation, Inspections (EB)& Lau:9, INSP & Tue 15-17\\
& E5 & Validation & Lau:9, INSP  & Thu 10-12 or 13-15\\
& P & Release R2 \newline send also to validation group & & Deadline Sun 23.59 \\
& P & Validation Checklist \newline send also to validation group & & Deadline Sun 23.59\\
\hline
\multirow{3}{*}{{\bfseries\sffamily W6}} 
& P &  Validation Report\newline send also to authoring group & & Deadline Thu 23.59  \\
\hline
\multirow{1}{*}{{\bfseries\sffamily W7}} 
& P & Supervision meeting & & \LASTSUPERVISION \\
\CONF
& P & Release R3 & & Deadline Sun 23:59\\
\hline
\multirow{1}{*}{{\bfseries\sffamily  }} 
& Exam & &All literature  & March 18, Fr 8-13\\
\multirow{1}{*}{{\bfseries\sffamily Mar }}   
&  P & \multirow{1}{*}{Course Evaluation -> BR EB, 1 per project} & & March 25, Fr 23.59\\
\hline
\end{tabular} 
\end{flushleft}

\ifteknolog\else 
\STARTFRG NOTE: Lecture L6 is optional for TFRG55.
\fi

\section{Personnel}
\begin{flushleft}
	\setlength{\tabcolsep}{0pt}
	\begin{tabular}{p{0.3\columnwidth} p{0.07\columnwidth} p{0.6\columnwidth}}
		Bjorn.Regnell & BR & Coordinator, Lectures, Exam \\
		Elizabeth.Bjarnason & EB & Coordinator, Lecture L5, Exercises, Exam \\
		Johan.Linaker & JL & Lecture L6\\
		Sergio.Rico & SR & Project and lab \\
		Ulrika.Templing & & Course admin \\
		Birger.Swahn & & Course admin \\
		...all e-mails @cs.lth.se\\
		Linnea Allander & LA & Project and lab \\
		li6687al-s@student.lu.se\\
	\end{tabular}
\end{flushleft}

\section{Where is the teaching?}
%Due to the pandemic we may have to, with short notice, transfer to online teaching, so check TimeEdit and notifications in Canvas and before each teaching event.

% \vspace{1em}
\noindent Preliminary room allocation -- double-check in TimeEdit for late changes:
\begin{flushleft}
\small
\begin{tabular}{l | l } 
{\it What} & {\it Where} \\
\hline
Lectures & E:C %with Zoom option, but L2, L5 Zoom only. 
\\
Exercises & E:3319 %Zoom option for those with symptoms. 
\\
LAB& E:Varg % Extra slots given later in Zoom for those with symptoms.
\\
Exam & MA10DE\\
\end{tabular}
\end{flushleft}

%Zoom-links for online teaching: see Canvas. 