%!TEX encoding = UTF-8 Unicode

\documentclass{program}

\newcommand{\COURSECODE}{ETSN15}
\newcommand{\COURSETITLE}{Requirements Engineering}
\newcommand{\STARTFRG}{}

\title{\bf\sffamily\fontsize{18}{18}\selectfont{
Course Program\\
\COURSECODE~\COURSETITLE\\
http://cs.lth.se/krav 
}}

\author{\bf\sffamily\fontsize{11}{11}\selectfont{Kursansvarig: Björn Regnell}}

\date{\bf\sffamily\fontsize{10}{10}\selectfont%
{Study period: 2025-VT1, Revision date: \today}}

\begin{document}

\maketitle
\noindent 
The objective of the course is to give basic and advanced knowledge and skills within requirements engineering for large-scale development of systems completely or partly based on software. The course gives both theoretical knowledge and practical skills in methods and techniques for requirements engineering. The course gives training in scientific paper reading.

\section{Learning Objectives}
\subsection{Knowledge and understanding}
For a passing grade the student must
\begin{enumerate}[noitemsep]
\item be able to define basic concepts and principles within requirements engineering 
\item give an account of several different types of requirements
\item be able to describe and value several different methods and techniques for requirements engineering
\item be able to describe and relate different sub-processes within requirements engineering
\item be able to describe the relation between the requirements engineering process and other processes in the product lifecycle
\item be able to describe the relation between requirements engineering and market-driven product management
\item be able to discuss some scientific results within requirements engineering research
\end{enumerate}


\subsection{Skills and abilities}
For a passing grade the student must
\begin{enumerate}[noitemsep]
\item be able to choose suitable requirements techniques for a given context
\item     be able to apply several different techniques for requirements elicitation
\item     be able to apply several different techniques for requirements specification
\item     be able to apply several different techniques for requirements validation
\item     be able to apply several different techniques for requirements prioritisation
\end{enumerate}
  The release plan defines which requirements that are implemented by the project group as mock-up designs in release R3, and which requirements are selected to be fully implemented in the imagined releases R4 and R5. 
\subsection{Judgement and approach}
For a passing grade the student must
\begin{enumerate}[noitemsep]
\item     be able to consciously select a process depending on the nature of the requirements
\item     show a systematic and long-term approach to processes
\item     be able to consciously see the problem in the relation between the quality of requirements and the quality of the resulting implementation
\item     be able to adequately involve users in the requirements engineering process
\item     be able to consciously see the problem in the relation between requirements engineering and economical aspects of product development
\end{enumerate}

\section{Contents}
The course includes theory and practice regarding the following topics:
\begin{enumerate}[noitemsep]
\item Requirements on different abstraction levels and in different contexts

\item Sub-processes of requirements engineering and their relation

\item Specification of data requirements, e.g. using virtual windows and data models

\item Specification of functional requirements, e.g. using textual feature requirements and task descriptions

\item Specification of different types of non-functional requirements, e.g. usability, performance, reliability

\item Different techniques for requirements elicitation

\item Different techniques for requirements validation

\item Different techniques for requirements prioritization

\item Market-driven requirements engineering and product management
\end{enumerate}

\newpage

\section{Course elements}
\begin{description}
\item[L: Lectures] The lectures provide an overview of the literature. They do not cover every detail, but give a high-level structure of the subject and thereby aid self-studies of the literature. Discussions are promoted.
\item[E: Exercises] The main objective of the exercises is to support the project and prepare for the written exam through prototypical problems, by connecting theory to practice and to give opportunity to discuss details of RE techniques.
\item[Lab: Computer lab sessions] The lab sessions illustrate computer supported
prioritization and release planning, and demonstrates the complexity of requirements selection and scheduling. Preparations are mandatory. %
\item[P: Project] The project involves performing practical requirements engineering for a given case, and is carried out in groups of 6-8 students. The project involves a number of deliverables and a final project conference where the learning outcome of each project is presented. Project groups are established during the first course week.

\end{description}




\section{Assessment}
\begin{itemize}
\item The project is graded fail | 3 | 4 | 5 based on project deliverables.
\item Approved lab session preparations and assignments are required for passing.
\item The written exam comprises of  multiple-choice questions, short essays and a practical assignments; in total 100 points of which 50p is required for passing. %50 percent is required for each part for passing. The essay and practical assignment part yields max 100 points and determines the grade fail | 3 | 4 | 5.
\item The final course grade on the scale fail | 3 | 4 | 5 is based on the written exam points and the project grade using the following mapping: 

\vspace{1em}

\begin{tabular}{r | c c c}
 & Project: 3 & Project: 4 & Project: 5 \\
\hline
 & \multicolumn{3}{c}{Exam points}    \\
Final: 3 & $ \geq 50$ & $\geq 50$ & $\geq 50$ \\
Final: 4 & $ \geq 75$ & $\geq 67$ & $\geq 60$ \\
Final: 5 & $ \geq 90$ & $\geq 83$ & $\geq 75$ \\
\hline
\end{tabular}


\end{itemize}

\newpage

\section{Literature}
The course elements and the written exam cover the following literature: 
\begin{flushleft}
\setlength{\tabcolsep}{0pt}
\begin{tabular}{p{0.15\columnwidth} p{0.85\columnwidth}}
Lau & ''Software Requirements - Styles and Techniques'', Soren Lauesen, Addison-Wesley, ISBN 0-201-74570-4, 2002. \\
LAB1\&2	&Preparations and instructions for Lab 1 and Lab 2. \newline \url{https://cs.lth.se/krav/labs/}\\
MDRE &	''Market-Driven Requirements Engineering for Software Products'', Björn Regnell, Sjaak Brinkkemper, Engineering and Managing Software Requirements, Eds. A. Aurum, C. Wohlin, Springer,  ISBN 3-540-25043-3, 2005.\\
PRIO&	''Requirements Prioritization'', Patrik Berander, Anneliese Andrews, Engineering and Managing Software Requirements, Eds. A. Aurum, C. Wohlin, Springer,  ISBN 3-540-25043-3, 2005. \\
OSSRE & ''Understanding Requirements for Open Source Software'', Walt Scacchi, Design Requirements Engineering: A Ten-Year Perspective, Springer, pp. 467-494, 2009.\\
RP&	''The Art and Science of Software Release Planning'', Günther Ruhe, Moshood Omolade Saliu, IEEE Software, November/December, pp. 47-53, 2005. \\
QUPER&	''Supporting Roadmapping of Quality Requirements'', Björn Regnell, Richard Berntsson Svensson, 
Thomas Olsson, IEEE Software 25(2) pp. 42-47 March-April 2008. \\
PROTO1& ''An empirically based model of software prototyping: a mapping study and a multi-case study'', Elizabeth Bjarnason, Franz Lang, Alexander Mjöberg, Empirical Software Engineering, 28(5), 115, 2023.\\
PROTO2& ''Prototyping practices in software startups: Initial case study results'', Elizabeth Bjarnason, IEEE 29th International Requirements Engineering Conference Workshops (REW), 2021.\\
INSP&	''Att inspektera krav''. Sid 67-76, Framgångsrik kravhantering, andra utgåvan, Teknikföretagen, Joachim Karlsson, V040072, ISSN 1103-7008, 1998.\\
INTDEP &	''An industrial survey of requirements interdependencies in software product release planning'', Pär Carlshamre, Kristian Sandahl, Mikael Lindvall, Björn Regnell, Johan Natt och Dag, Int. Conf. on Requirements Engineering (RE01), Toronto, Canada, pp. 84–91, 2001.\\
AGRE &	''Agile Requirements Engineering Practices: An Empirical Study'', Lan Cao, Balasubramaniam Ramesh, IEEE Software , January/February 2008, pp. 60-67, 2008.\\
\end{tabular}
\end{flushleft}

\newpage

\section{Overview}
%L = Lecture, E = Exercise, P = Project, Lab = Computer Lab
%% IT IS NOT ALLOWED TO HAVE CONDITIONALS INSIDE tabular
%% so adding this newcommand, which is interpreted before the below tabular
\newcommand{\CONF}{
& P & Conference presentation (pdf) & & Deadline Wed 8:00  \\
& P & Project conference (mandatory) &  & Mon 15-17\\
}

\newcommand{\LASTSUPERVISION}{book with supervisor on \newline Mon or Tue (or W6) }

\newcommand{\LABHRS}{Thu or Fri; see Canvas}

\begin{flushleft}
\small
\begin{tabular}{c | p{0.6cm} p{4.4cm} p{2.2cm}  p{3.1cm}}
 &  & {\it Topic} & {\it Literature} & {\it When}   \\
\hline
\multirow{4}{*}{{\bfseries\sffamily W1}} 
& L1& Introduction  & Lau:1 & Mon 15-17\\
& L2 & Project kick-off, Elicit, Prio  & Lau:8, PRIO  & Tue 15-17\\
& P& Project Mission v1  &  & Deadline Wed 10:15\\
& E1 & Requirements types, \newline Context diagram &  Lau:1  & Wed 10-12\\
\hline
\multirow{4}{*}{{\bfseries\sffamily W2}} 

& L3& Specification 1, reqT    & Lau: 2-4  & Mon 15-17\\
& L4& Specification 2  &  & Tue 15-17 \\
& P  & Project Mission v2&  & Tue 23:59 \\
& P & Supervision meeting & & Contact supervisor\\
& E2 & Elicitation  & Lau: 8  & Wed 10-12\\
& Lab1 & Reqts Modelling, Prio & LAB1 &   \LABHRS\\
\hline
\multirow{4}{*}{{\bfseries\sffamily W3}} 

& L5 \newline & Agile RE, \newline Prototyping,  
& AGRE, PROTO1-2, & Mon 15-17 \\

& L6\newline  & Product mgmt, Release planning, Open source RE, Interdep. & MDRE, RP, \newline OSSRE, \newline INTDEP & Tue 15-17 \\

& E3 & Functional reqts, Prototyping  & Lau:2-4  & Wed 10-12\\
& P & Release R1& & Deadline Sun 23:59 \\
\hline
\multirow{3}{*}{{\bfseries\sffamily W4}} 
& L7 & Specification 3, Quality, Lifecycle & Lau:5-7, QUPER  & Mon 15-17\\
& E4 & Quality requirements &  Lau:6, QUPER  &Wed 10-12\\
& P & Supervision meeting & & Contact supervisor\\
& Lab2 & Prioritization, Release Planning & LAB2 &  \LABHRS\\
\hline
\multirow{4}{*}{{\bfseries\sffamily W5}} 
& L8 & Validation, exam prep&  Lau:9, INSP & Mon 15-17\\
& E5 & Validation & Lau:9, INSP  & Wed 10-12\\
& P & Release R2 & & Deadline Sun 23:59 \\
& P & Validation Checklist \newline \textit{send to validation group} & & Deadline Sun 23:59\\
\hline
\multirow{3}{*}{{\bfseries\sffamily W6}} 
& P & Supervision meeting (W6 or W7)& &  Contact supervisor\\
& P &  Validation Report\newline \textit{send to authoring group} & & Deadline Thu 23:59  \\
\hline
\multirow{1}{*}{{\bfseries\sffamily W7}} 
\CONF
& P & Release R3 & & Deadline Sun 23:59\\
\hline
\multirow{1}{*}{{\bfseries\sffamily  }} 
& Exam & &All literature  & March 15, 8:00-13:00\\
\multirow{1}{*}{{\bfseries\sffamily Mar }}   
&  P & \multirow{1}{*}{Course Evaluation -> BR, 1 per project} & & March 27, 23:59\\
\hline
\end{tabular} 
\end{flushleft}


\section{Personnel}
\begin{flushleft}
	\setlength{\tabcolsep}{0pt}
	\begin{tabular}{p{0.3\columnwidth} p{0.09\columnwidth} p{0.6\columnwidth}}
		Bjorn.Regnell & BR & Coordinator, Lectures \\
		Elizabeth.Bjarnason & EB & Exercises, Exam \\
		Matthias.Wagner & MW & Project and Labs \\
		Birger.Swahn & & Course admin \\
		Ulrika.Templing & & Course admin \\
		...all e-mails @cs.lth.se\\
	\end{tabular}
\end{flushleft}

\section{Where is the teaching?}
%Due to the pandemic we may have to, with short notice, transfer to online teaching, so check TimeEdit and notifications in Canvas and before each teaching event.

% \vspace{1em}
\noindent Preliminary room allocation -- double-check in TimeEdit for late changes:\\ \url{https://cs.lth.se/krav/schema/}
\begin{flushleft}
\small
\begin{tabular}{l | l } 
{\it What} & {\it Where} \\
\hline
Lectures & E:C  
\\
Exercises & M:Ångpannehallen %Zoom option for those with symptoms. 
\\
Labs & E:Ravel % Extra slots given later in Zoom for those with symptoms.
\\
Exam & E:3308\\
\end{tabular}
\end{flushleft}

%Zoom-links for online teaching: see Canvas. 

\end{document}
